\chapter{left-shift binary algorithm}
\section{Definición}
Este algoritmo debe su nombre al hecho de que se hace multiplicación por 2. En representación
binaria el efecto de multiplicar por dos es un desplazamiento (en una posición), de la representación
binaria original, hacia la izquierda.

\section{Implementacion del algoritmo}
\begin{lstlisting}[language=C++]
#include <iostream>
#include <cmath>
#include <ctime>
#include <fstream>
#include <NTL/ZZ.h>
using namespace std;
using namespace NTL;
unsigned t0,t1;
void guardar(ZZ a,ZZ b,ZZ times,ZZ aux, ZZ s)
{
    ofstream archivo;
    archivo.open("lsbgcd.csv",ios::app);
    archivo << a <<"&"<< b<<"&"<<times<<"&"<< aux<<"&"<<s << "\\\\\\hline"<< endl;
    archivo.close();
}

ZZ menor(ZZ x,ZZ y)
{
    if(y<x)
        return y;
    else 
        return x;
}
ZZ potencia(ZZ a,ZZ b)
{
    ZZ result;
    result=1;
    for (int i = 0; i < b; i++)
    {
         result=a*result;
    }
    return result;
}
ZZ  lsbgcd(ZZ u, ZZ v)
{
    ZZ a,b,t,aux,s,two;
    two=2;
    a=abs(u);
    b=abs(v);
    cout << "a" << '\t'<< '\t' << "b" << '\t'<< '\t' << "t" << '\t'<< '\t' << "aux" << '\t'<< '\t' << "s" << endl;
    while(b!=0)
    {
        s=0;
        while(b*potencia(two,s)<=a)
        {
            s=s+1;
        }
        s=s-1;
        t=menor(a-b*potencia(two,s),b*potencia(two,s+1)-a);
        a=b;
        b=t;
        if(a<b)
        {
            aux=a;
            a=b;
            b=aux;
        }
        cout << a << '\t'<< '\t' << b << '\t'<< '\t' << t << '\t' << '\t'<< aux << '\t'<< '\t' << s <<endl; 
        guardar(a,b,t,aux,s);
    }   
    return a;
}
//save data 
int main()
{
    ZZ a,b;
    cout << "input a:" ; cin >> a; //imput the numbers 
    cout << "input b:" ; cin >> b;
    lsbgcd(a,b);   
}
\end{lstlisting}
\section{Seguimiento del codigo}
\begin{table}[H]
\label{tablax}
\begin{center}
\begin{tabular}{|c|c|c|c|c|}
\hline 
a&b&t&aux&s \\
\hline
51459615586&94237394&51459615586&94237394&13\\\hline
3210069858&94237394&3210069858&94237394&9\\\hline
194473250&94237394&194473250&94237394&5\\\hline
94237394&5998462&5998462&94237394&1\\\hline
5998462&1737998&1737998&94237394&3\\\hline
1737998&953530&953530&94237394&1\\\hline
953530&169062&169062&94237394&0\\\hline
277282&169062&277282&169062&2\\\hline
169062&60842&60842&169062&0\\\hline
60842&47378&47378&169062&1\\\hline
47378&13464&13464&169062&0\\\hline
13464&6478&6478&169062&1\\\hline
6478&508&508&169062&1\\\hline
1650&508&1650&508&3\\\hline
508&382&382&508&1\\\hline
382&126&126&508&0\\\hline
126&122&122&508&1\\\hline
122&4&4&508&0\\\hline
6&4&6&4&4\\\hline
4&2&2&4&0\\\hline
2&0&0&4&1\\\hline
\end{tabular}
\end{center}
\caption{seguimiento de codigo}
\end{table}
