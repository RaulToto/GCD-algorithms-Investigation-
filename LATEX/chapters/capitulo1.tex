\chapter{Algoritmo de Menor Resto}
\section{Definición}
es muy apropiada para fines teóricos. Que el resto sea
positivo es adecuado, como vimos, para mostrar unicidad.
Sin embargo el resto no tiene porque ser positivo, por ejemplo si a = 144 y b = 89,
\begin{itemize}
    \item 144 = 89 · 1 + 55, resto r2 = $55 < b = 89$
    \item 144 = 89 · 2 − 34, resto r1 = $34 < b = 89$

\end{itemize}

\section{Implementacion del algoritmo}
\begin{lstlisting}[language=C++]
#include <iostream>
#include <NTL/ZZ.h>//includ  the lib ntl 
#include <ctime>
#include <fstream>
using namespace std;//using std namespace 
using namespace NTL;//using NTL namespaces
unsigned t0,t1;

ZZ module(ZZ &x,ZZ &y){
    ZZ q=x/y,r;
    if(q<0)
    {
        q=-1*q;
        q++;
        q=-1*q;
        r=x-(q*y);
    }
    else 
    {
        r=x-(q*y);
    }
    return r;
}
ZZ menor(ZZ x,ZZ y)
{
    if(y<x)
        return y;
    else 
        return x;
}
void euclides(ZZ a, ZZ b)//
{
    //t0=clock();
    ZZ q,q1,r;
    q=a/b;
    q1=q;q1++;
    r=menor(a-(q*b),a-(q1*b));
    //r=a%b;
    while(r!=0)
    {
        q=a/b;
        q1=q;q++;
        r=menor(a-(q*b),a-(q1*b));
        //cout << a  << '\t'<<  " = " << q << "(" << b << ")" << "+" << r << endl;//print the euclides algorithm
        a=b;
        b=r;
    }
    cout << "result is "<< a << endl;
    //t1=clock();
}
//funcion para guardar los datos 
int main()
{
    ZZ a,b;
    cout << "input a:" ; cin >> a; //imput the numbers 
    cout << "input b:" ; cin >> b;
    euclides(a,b);
    //double time=(double(t1-t0)/CLOCKS_PER_SEC);
    //cout << "Execution time:" << time << endl;

}
\end{lstlisting}
\section{Seguimiento del codigo}
\begin{table}[H]
\label{tablax}
\begin{center}
\begin{tabular}{|c|c|c|c|}
\hline 
a&b&q&r \\
\hline
9792347293422342&10048&974592347234&-356611584890\\\hline
974592347234&-2&-356611584890&-95242407436\\\hline
-356611584890&4&-95242407436&-70884362582\\\hline
-95242407436&2&-70884362582&-24358044854\\\hline
-70884362582&3&-24358044854&-22168272874\\\hline
-24358044854&2&-22168272874&-2189771980\\\hline
-22168272874&11&-2189771980&-270553074\\\hline
-2189771980&9&-270553074&-25347388\\\hline
-270553074&11&-25347388&-17079194\\\hline
-25347388&2&-17079194&-8268194\\\hline
-17079194&3&-8268194&-542806\\\hline
-8268194&16&-542806&-126104\\\hline
-542806&5&-126104&-38390\\\hline
-126104&4&-38390&-10934\\\hline
-38390&4&-10934&-5588\\\hline
-10934&2&-5588&-5346\\\hline
-5588&2&-5346&-242\\\hline
-5346&23&-242&-22\\\hline
-242&12&-22&0\\\hline
\end{tabular}
\end{center}
\caption{seguimiento de codigo}
\end{table}