\chapter{Algoritmo Lehmer GCD}

\section{Algoritmo}
El algoritmo de MCD extendido de Lehmer D. H.
Lehmer propuso en un algoritmo dirigido a optimi-
zar el cálculo del máximo común divisor (MCD) de
dos enteros positivos de múltiple precisión utilizan-
do mayormente operaciones de precisión simple, lo
que permite utilizar en la mayor parte de los casos
operaciones internas del procesador. Este algoritmo
puede extenderse de manera directa a un algoritmo
de MCD extendido, aplicable para encontrar la in-
versa modular de sólo un dı́gito. Además, observó
que el proceso subyacente en el algoritmo de Eucli-
des es la aplicación de sucesivas transformaciones
lineales:
\begin{itemize}
 \item[1] While y $\geq$ b do the following:
 \begin{itemize}
  \item[1.1] Set x\_, y\_ to be the high-order digit of x, y, respectively (y\_ could be 0)
  \item[1.2] $A\leftarrow1$, $B\leftarrow0$, $C\leftarrow0$, $D\leftarrow1$ 
  \item[1.3] If ($grupos\_x == grupos\_y$) esto se agregó al algoritmo del presente libro 
  \item[1.4] While ($y\_ + C$) $!= 0$ and $(y\_ + D)!= 0$ do the following:
  \begin{itemize}
   \item $q\leftarrow (x\_ + A)/(y\_ + C)$ , $q\_ \leftarrow (x\_ + B)/(y\_+ D)$ 
   \item if $q = q\_$ then go to step 1.5
   \item $t \leftarrow A-qC, A \leftarrow C, C \leftarrow t, t\leftarrow B - qD, B\leftarrow D, D\leftarrow t$ 
   \item $t\leftarrow x\_ - qy\_, x\_\leftarrow y\_, y\_\leftarrow t$
  \end{itemize}
  \item[1.5] If $B = 0$, then $T \leftarrow x mod y$, $x\leftarrow y, y\leftarrow T$ 
  \item otherwise, $T\leftarrow Ax + By, u\leftarrow Cx + Dy, x\leftarrow T , y\leftarrow u$ \\
 \end{itemize}
 \item[2] Compute $v = gcd(x, y)$ using Algorithm 2.104 
 \item[3] Return(v)
\end{itemize}

\section{Implementaci\'on}
\begin{lstlisting}[language=C++]
#include <iostream>
#include <limits.h>
#include <NTL/ZZ.h>
#include <fstream>
#define MAX_POTENCIAS 3 // para generar un arreglo de potencias de la base
#include <module.h>//this is my library Module 
using namespace std;
using namespace NTL;
void lehmer_gcd(ZZ x, ZZ y);
ZZ enuclides( ZZ a,  ZZ b);
void guardar(int a,int b,int times,int resultado,int x,int y)
{
    ofstream archivo;
    //string line = "\\\\hline";
    archivo.open("lehmer.csv",ios::app);
    archivo << a <<"&"<< b<<"&"<<times <<"&" << resultado<<"&" << x <<"&" << y <<  "\\\\\\hline"<<endl;
    archivo.close();
}
int main()
{
    double segs;
    ZZ x = 565642334;
    ZZ y = 24423424;

    cout << "Aplicando Lehmer, gcd(" << x << ',' << y << ")\n";
    lehmer_gcd(x,y);

    int dj;
    clock_t t_ini = clock();
    dj = euclides(x,y);
    clock_t t_fin = clock();
    cout << dj << " en " << (double)(t_fin - t_ini)*1000.0 / CLOCKS_PER_SEC;
    return 0;

}

ZZ *genera_array_base(ZZ base);
ZZ digitos_base(ZZ num, ZZ arr[], ZZ &grupos);

void lehmer_gcd(ZZ x, ZZ y)
{
    clock_t t_ini = clock();

    ZZ x_, y_, a, b, c, d;
    ZZ q, q_, t, tt, u;
    ZZ grupos_x;
    ZZ grupos_y;
    ZZ base = 1000; // determina un arreglo de potencias de la base.
    ZZ length_bits = sizeof(int)*CHAR_BIT;
    ZZ *arr_potencias = genera_array_base(base); // se crea un array con las potencias de la base.
                                       // si es base 2 se trata de una forma especial (moviendo bits)
    while(y >= base){

        x_ = digitos_base(x, arr_potencias, grupos_x ); // x_ tendrá los dígitos más significativos que unidos serán <= a la base.
                              // la función usa búsqueda binaria en el array de potencias  de la base
        y_ = digitos_base(y, arr_potencias, grupos_y);
	
        a = 1; b = 0; c = 0; d = 1;
        if(grupos_x == grupos_y){ //necesario pues en la siguiente vuelta hay que asegurar que x e y tengan la misma cantidad d cifras
            while( ((y_+c) != 0) && ((y_+d) != 0) ){
                q = (x_+a) / (y_+c);
                q_ = (x_+b)/ (y_+d);
                if (q != q_){
                    break;
                }
                t=a-q*c;
                a = c;
                c = t;
                t = b - q*d;
                b = d;
                d = t;
                t = x_ - q*y_;
                x_ = y_;
                y_ = t;
                guardar(x_,y_,a,b,c,d);
            }
            
        }
        if (b == 0){
            tt = x%y;
            x = y;
            y = tt;
        }
        else{
            tt = a*x + b*y;
            u = c*x + d*y;
            x = tt;
            y = u;
        }
    }

    clock_t t_fin = clock();

    cout << "Lehmer redujo a gcd(" << x << ',' << y << ") en ";
    double segs = (double)(t_fin - t_ini) / CLOCKS_PER_SEC;
    cout << segs * 1000.0 << " milisegundos. \n" << '\n';

    cout << "euclide  gcd(" << x << ',' << y << ") = ";

    t_ini = clock();
    ZZ v = euclides(x,y);
    t_fin = clock();
    cout << v << '\n';
    segs += (double)(t_fin - t_ini) / CLOCKS_PER_SEC;
    cout << "\nTiempo total:" << segs * 1000.0 << " milisegundos" << '\n';
}

ZZ *genera_array_base(ZZ base)
{
    //repetir el primer y último elemento para
    //  retornar lo deseado en la busq. binaria
    //  así se agregan dos elementos más
    ZZ *arr = new ZZ[MAX_POTENCIAS+2];

    arr[0] = base;
    ZZ potencia = 1;
    for(ZZ i = 1; i != MAX_POTENCIAS+1; i++){
        potencia *= base;
        arr[i] = potencia;
    }
    arr[MAX_POTENCIAS+1] = potencia;

    return arr;
}

ZZ b_binaria(ZZ num, ZZ arr[], ZZ low, ZZ high); // devuelve el valor de la
                                                    // potencia <= num dentro de arr
ZZ digitos_base(ZZ num, ZZ arr[], ZZ &grupos)
{
    grupos = b_binaria(num, arr, 0, MAX_POTENCIAS+2 -1);
    if (grupos == 0)
        grupos = 1;
    else
        if (grupos == MAX_POTENCIAS+1 )
            grupos = MAX_POTENCIAS;
    return num / arr [grupos];
}
ZZ b_binaria(ZZ x, ZZ arr[], ZZ low, ZZ high)
{
    ZZ medio;
    if (high > low)
        medio= (high-low)/2 + low;
    else
        medio = (low-high)/2 + low;
    if(arr[medio] == x || (low > high))
        return (low-1); //medio == low, asi retorna la potencia menor de la base
    else{
        if (arr[medio] < x)
            return b_binaria(x,arr,medio+1, high);
        else
            return b_binaria(x,arr,low,medio-1);
    }
}
ZZ euclides(ZZ a, ZZ b)//
{
    ZZ q,r;
    q=a/b;
    r=module(a,b);
    //r=a%b;
    while(r!=0)
    {
        q=a/b;
        r=module(a,b);
        a=b;
        b=r;
               
    }

    return b;
}
\end{lstlisting}
\section{seguimiento del codigo}
\begin{table}[H]
\label{tablax}
\begin{center}
\begin{tabular}{|c|c|c|c|c|c|}
\hline 
x&y&a&b&c&d \\
\hline
104&65&0&1&1&-8\\\hline
65&39&1&-8&-1&9\\\hline
39&26&-1&9&2&-17\\\hline
25&14&0&1&1&-1\\\hline
14&11&0&1&1&-1\\\hline
11&3&1&-1&-1&2\\\hline
\end{tabular}
\end{center}
\caption{seguimiento de codigo}
\end{table}

      
      
    
      

