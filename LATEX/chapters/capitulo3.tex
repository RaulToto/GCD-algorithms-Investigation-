\chapter{Algoritmo Binario de Euclides}
\section{Definición}
El algorirmo binario de GCD , tambien conociddo  como algortimo de stein es un algoritmo que calcula el mayor divisor común de dos enteros no negativos.
El algoritmo de Stein utiliza operaciones aritméti-
cas más simples que el algoritmo euclidiano conven-
cional; Reemplaza la división por cambios aritméti-
cos, comparaciones y sustracciones. Aunque el al-
goritmo fue publicado por primera vez por el fı́sico
y programador israelı́ Josef Stein en 1967, [2] puede
haber sido conocido en la China del siglo I.
Opera con los siguientes teoremas
\section{el algoritmo de binario gcd}
\begin{itemize}
 \item Si a,b son pares, $mcd(a,b) = 2 mcd(a/2,b/2)$.
 \item Si a es par y b impar o viceversa, $mcd(a,b) = mcd(a/2,b)\: o\: mcd(a, b/2)$.
 \item Si a,b son impares , $mcd(a,b) = mcd(|a-b|/2, r)$, donde $r = min{a,b}$;
\end{itemize}
\section{Implementación}
\begin{lstlisting}[language=C++]

ZZ binary_gcd( ZZ u,ZZ v)
{
  int i;
  if (u == 0) return v;
  if (v == 0) return u;
  cout << "a" << '\t' << "b" <<'\t' << "i" <<  endl;
  for (i = 0; ((u | v) & 1) == 0; ++i) {
         u >>= 1;
         v >>= 1;
        //cout << u << '\t' << v << '\t' << i <<  endl;      
  }

  while ((u & 1) == 0)
    u >>= 1;
    
  do {
       //cout << u << '\t' << v <<'\t' << i <<  endl;
       while ((v & 1) == 0)  
           v >>= 1;
       if (u > v) {
          ZZ t = v; v = u; u = t;}  
       v = v - u;                       
     } while (v != 0);
  return u << i;
}
\end{lstlisting}
\section{Seguimiento del algoritmo}
\begin{table}[H]
\label{tablax}
\begin{center}
\begin{tabular}{|c|c|c|}
\hline 
a&b&t \\
\hline
843563845&34534&0\\\hline
17267&843546578&0\\\hline
17267&421756022&0\\\hline
17267&210860744&0\\\hline
17267&26340326&0\\\hline
17267&13152896&0\\\hline
17267&85490&0\\\hline
17267&25478&0\\\hline
12739&4528&0\\\hline
283&12456&0\\\hline
283&1274&0\\\hline
283&354&0\\\hline
177&106&0\\\hline
53&124&0\\\hline
31&22&0\\\hline
11&20&0\\\hline
5&6&0\\\hline
3&2&0\\\hline
1&2&0\\\hline
\end{tabular}
\end{center}
\caption{seguimiento de codigo}
\end{table}



