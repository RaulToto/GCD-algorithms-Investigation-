\chapter{Algoritmo de Euclides clasico}

\section{Definici\'on}
El algoritmo de Euclides es un m\'etodo antiguo y eficaz para calcular el m\'aximo com\'un divisor (MCD). Fue originalmente descrito por Euclides en su obra Elementos.este algoritmo consiste en : \\
1.Si b=0 entonces maximoComunDivisor(a,b)=a y termina el algortimo\\
2.En otro caso maximoComunDivisor(a,b)=maximoComunDivisor(b,r) donde r es el resto al dividir a entre b.Para calcular maximoComunDivisor(b,r) se utilizan las mismas reglas \\

\section{Algoritmo}
Recordemos que $\:mod(a, b)$ denota el resto de la división de a por b. En este algoritmo, en cada paso $\:r = mod (rn+1, rn)$ donde $\:rn+1 = c$ es el dividendo actual y $\:rn = d$ es el divisor actual.
Luego se actualiza $\:rn+1 = d$ y $d = r$. El proceso continúa mientras d no se anule.\\
Datos: $\:a,b \in Z\:/b\neq0$\\
Salida: $mcd(a,b)$\\
\begin{equation}
 \begin{align}
  c=&|a|,d=|b|;\\
  while&\:d\neq0\:do\\
  r&=mod(c,d);\\
  c&=d;\\
  d&=r;\\
  return&\:mcd(a,b)=|c|;
 \end{align}
\end{equation}
\section{Seguimiento del codigo}
\begin{table}[H]
\label{tablax}
\begin{center}
\begin{tabular}{|c|c|c|c|}
\hline 
a&b&q&r \\
\hline
957349573453465&8346583456&114699&4797633721\\\hline
8346583456&4797633721&1&3548949735\\\hline
4797633721&3548949735&1&1248683986\\\hline
3548949735&1248683986&2&1051581763\\\hline
1248683986&1051581763&1&197102223\\\hline
1051581763&197102223&5&66070648\\\hline
197102223&66070648&2&64960927\\\hline
66070648&64960927&1&1109721\\\hline
64960927&1109721&58&597109\\\hline
1109721&597109&1&512612\\\hline
597109&512612&1&84497\\\hline
512612&84497&6&5630\\\hline
84497&5630&15&47\\\hline
5630&47&119&37\\\hline
47&37&1&10\\\hline
37&10&3&7\\\hline
10&7&1&3\\\hline
7&3&2&1\\\hline
3&1&3&0\\\hline
\end{tabular}
\end{center}
\caption{seguimiento de codigo}
\end{table}


\section{Implementacion del algoritmo}
\begin{lstlisting}[language=C++]
ZZ euclides(ZZ a, ZZ b)//
{
    ZZ q,r;
    q=a/b;
    //r=module(a,b);
    r=a%b;
    while(r!=0)
    {
        q=a/b;
        //r=module(a,b);
        r=a%b;
        a=b;
        b=r;
    }
    return r;
}
\end{lstlisting}
